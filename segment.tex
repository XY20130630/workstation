\documentclass[a4paper]{article}
\usepackage{ctex}
\usepackage{xeCJK}
\usepackage{amsmath}
\usepackage{amsfonts}
\usepackage{amssymb}
\usepackage{graphicx}
\usepackage{colortbl}
\usepackage{fancyvrb}
\usepackage{longtable}
\usepackage{xcolor}
\usepackage[hidelinks]{hyperref}
\usepackage[affil-it]{authblk}
\usepackage[top = 1.0in, bottom = 1.0in, left = 1.0in, right = 1.0in]{geometry}
\usepackage{amsthm}

\setCJKfamilyfont{kai}{KaiTi_GB2312}
\newcommand{\kai}{\CJKfamily{kai}}                            

\newcommand\spc{\vspace{6pt}}
\newcommand{\floor}[1]{\lfloor {#1} \rfloor}
\newcommand{\ceil}[1]{\lceil {#1} \rceil}
\newcommand*\chem[1]{\ensuremath{\mathrm{#1}}}

\newtheorem{theorem}{Theorem}[section]
\newtheorem{lemma}[theorem]{Lemma}

\date{\today}
%\date{\yestoday}
\title{线段树部分总结}
\author{$\mathcal Pyh$}

\begin{document}

\maketitle

\kai

\tableofcontents

\newpage

\section{\kai{按位置建线段树的基础应用}}

有一些典型的数据结构题,即给出几种操作,动态询问区间的某些信息。下面我总结了几类按位置建线段树能够解决的难题。

\subsection{\kai{一类难点在区间合并与维护信息的线段树问题}}

有一类问题,初看到问题会无从下手,往往操作十分简单,但是询问的内容比较复杂。

\subsubsection{\kai{从简单情况开始分析}}

\subsubsection{\kai{分析题目性质}}

\subsection{\kai{一类难点在处理标记的线段树问题}}

\subsubsection{\kai{从全局出发,一种“操作log次就不再有效”的问题}}

\subsubsection{\kai{李超线段树}}

\subsubsection{\kai{区间最值操作与历史最值询问}}

\section{\kai{权值线段树的基础应用}}

\subsection{\kai{部分代替平衡树}}

\section{\kai{可持久化线段树的基础应用}}

\subsection{\kai{一类关于子序列的问题}}

\subsection{\kai{一类维护序列中某个区间信息的问题}}

\section{\kai{动态开点线段树的基础应用}}

\section{\kai{线段树合并的基础应用}}

\subsection{\kai{一类考虑子树对父亲的贡献的问题}}

\subsection{\kai{一类有关图的最短路的问题}}

\section{\kai{线段树套线段树的基础应用}}

\section{\kai{线段树作为辅助数据结构的一些问题}}

\subsection{\kai{用线段树优化dp}}

\subsection{\kai{用线段树判断完美匹配}}

\subsection{\kai{用线段树模拟费用流}}

\section{\kai{线段树问题的一些小技巧}}

\subsection{\kai{差分}}

\subsection{\kai{下标有关}}

\end{document}
