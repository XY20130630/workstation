\documentclass[a4paper]{article}
\usepackage{ctex}
\usepackage{amsmath}
\usepackage{amsfonts}
\usepackage{amssymb}
\usepackage{graphicx}
\usepackage{colortbl}
\usepackage{fancyvrb}
\usepackage{longtable}
\usepackage{xcolor}
\usepackage[hidelinks]{hyperref}
\usepackage[affil-it]{authblk}
\usepackage[top = 1.0in, bottom = 1.0in, left = 1.0in, right = 1.0in]{geometry}
\usepackage{amsthm}

\newcommand\spc{\vspace{6pt}}
\newcommand{\floor}[1]{\lfloor {#1} \rfloor}
\newcommand{\ceil}[1]{\lceil {#1} \rceil}
\newcommand*\chem[1]{\ensuremath{\mathrm{#1}}}

\newtheorem{theorem}{Theorem}[section]
\newtheorem{lemma}[theorem]{Lemma}

\date{\today}
%\date{\yestoday}
\title{线段树部分总结}
\author{$\mathcal Pyh$}

\begin{document}

\maketitle

\section{按位置建线段树的基础应用}

\subsection{一类难点在区间合并与维护信息的线段树问题}

\subsubsection{从简单情况开始分析}

\subsubsection{分析题目性质}

\subsection{一类难点在处理标记的线段树问题}

\subsubsection{从全局出发,一种“操作log次就不再有效”的问题}

\subsubsection{李超线段树}

\subsubsection{区间最值操作与历史最值询问}

\section{权值线段树的基础应用}

\subsection{部分代替平衡树}


\end{document}
