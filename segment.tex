\documentclass[a4paper]{article}
\usepackage{ctex}
\usepackage{xeCJK}
\usepackage{amsmath}
\usepackage{amsfonts}
\usepackage{amssymb}
\usepackage{graphicx}
\usepackage{colortbl}
\usepackage{fancyvrb}
\usepackage{longtable}
\usepackage{xcolor}
\usepackage[hidelinks]{hyperref}
\usepackage[affil-it]{authblk}
\usepackage[top = 1.0in, bottom = 1.0in, left = 1.0in, right = 1.0in]{geometry}
\usepackage{amsthm}

\setCJKfamilyfont{kai}{KaiTi_GB2312}
\newcommand{\kai}{\CJKfamily{kai}}

\setCJKfamilyfont{song}{SimSun}
\newcommand{\song}{\CJKfamily{song}}

\newcommand\spc{\vspace{6pt}}
\newcommand{\floor}[1]{\lfloor {#1} \rfloor}
\newcommand{\ceil}[1]{\lceil {#1} \rceil}
\newcommand*\chem[1]{\ensuremath{\mathrm{#1}}}

\newtheorem{theorem}{Theorem}[section]
\newtheorem{lemma}[theorem]{Lemma}
\newtheorem{problem}{例题}

\date{\today}
%\date{\yestoday}
\title{线段树部分总结}
\author{$\mathcal Pyh$}

\begin{document}

\maketitle

%\kai

\song

\tableofcontents

\newpage

\section{\kai{按位置建线段树的基础应用}}

有一些典型的数据结构题,即给出几种操作,动态询问区间的某些信息。下面我总结了几类按位置建线段树能够解决的难题。

\subsection{\kai{一类难点在区间合并与维护信息的线段树问题}}

有一类问题,初看到问题会无从下手,往往修改操作十分简单,但是询问的内容比较复杂,甚至某些问题完全没有修改操作,只有区间的动态询问。

下面我总结了一些解决这类问题的方法。

\subsubsection{\kai{从简单情况开始分析}}

这是一种通用的方法,我们可以首先从简单情况分析起,最后类比推理到复杂的情况。

\begin{problem}
  k-Maximum Subsequence Sum\footnote{Codeforces 280D}

  给出一个长度为n的序列A, m次操作,操作有如下两种:

  1.给出i,val,把$A_{i}$变成$val$。

  2.给出l,r,k,询问把区间$[l,r]$划分成不超过k个不相交的区间,这些区间中数的和的最大值。

  数据范围:$n,m\leq 10^5, k\leq 20$
\end{problem}

这里介绍一种$O(k^2(n+m)\log n)$的算法(下面将会介绍更优的算法)。

因为是单点修改,我们考虑怎么将两个子区间的信息合并到当前区间。

$k=1$ 的情况就是区间最大子段和,是一个经典问题,不再赘述。

考虑 $k=2$ 的情况。

我们发现选的两个子区间要么都不跨左右子区间的边界,要么一个跨了另一个没跨。

这就相当于在一个子区间中是最大后缀(前缀)和,在另一个子区间中是一端贴着边界,总共包含两个子区间的最大和。

这个相当好维护。那么能不能推广呢?

首先我们在每个子区间维护一个数组 $maxx[1..k]$,$maxx[i]$记录该区间划分成$i$个子区间的最大和。

显然首先当前区间划分成的 $k$ 段区间可以完全在左子区间和右子区间,直接统计答案即可。

然后假设选的所有区间不跨两个子区间的边界,我们可以枚举左子区间放了多少个区间,直接统计答案。

跨边界的情况怎么办呢?我们注意到跨边界的那个区间可以拆分成两个部分,一个在左子区间,一个在右子区间。

于是我们对于一个区间,维护一个信息,表示当前区间划分了 $k$ 个子区间,其中最右边的子区间是贴着右边界的最大和。

同理,维护一个最左边的子区间是贴着左边界的最大和(这些都是数组,因为一个子区间可能是放$1..k$中的任意多个区间)。

为了维护上面的这两个“贴着边界最大和”的信息,我们需要再次考虑跨过中点的情况:这次轻车熟路了,我们只需要维护两端都贴着边界的最大和即可。

而维护“两端都贴着边界最大和”的信息,我们可以直接通过左右子区间维护的上述信息维护出来。

因为计算维护的信息数组的每一项需要枚举左右区间各自放了多少个区间,所以区间合并的复杂度是$O(k^2)$的,整体的复杂度为$O(k^2(n+m)\log n)$。

这是一个很巧妙的算法,但是因为 $k^2$ 比较大而无法通过本题。下文我将会讲述这个题目的正确解法,也是使用了线段树,不过是作为辅助数据结构来使用的。

\subsubsection{\kai{分析题目性质}}

解决这类问题的另一种通用方法是分析题目中询问的信息的性质,来达到便捷地维护区间信息的目的。

\begin{problem}
  楼房重建\footnote{bzoj2957}

  在二维平面内,给出$m$个操作。

  每个操作给出 $x,y$, 把横坐标为$x$处的线段删除,并新建一条从 $(x,0)$ 到 $(x,y)$ 的线段,

  每个操作完成后询问在$(0,0)$能够看到多少条线段的顶部(线段可以互相遮挡)。

  数据范围:$m\leq 10^5$,横坐标$\leq 10^5$
\end{problem}

显然问题可以转化成满足下面这个条件的线段的个数:前面的所有线段的 $y/x$ 均严格小于它的$y/x$。

所以我们不妨把所有线段的高度变成$y/x$。

因为是单点修改,我们同样思考怎样合并两个子区间的信息。

假设当前考虑的区间的前面所有线段的高度最大值为$d$,

若左儿子的高度最大值$<d$,则左儿子的所有线段完全被挡住,左儿子的贡献为0,但是右儿子的贡献可能不为0,递归右儿子;

若左儿子的高度最大值$\geq d$,则左儿子的这个最大的线段完全挡住了左边的区间,在左边区间的修改完全不会影响到右儿子的线段,直接用右儿子在修改前对当前区间的贡献,而左儿子的线段可能会被影响,递归左儿子。

于是,在每次区间合并的过程中,我们要么递归左儿子要么递归右儿子,一次区间合并的复杂度为$O(\log n)$,总的复杂度为$O(n\log^2 n)$,可以通过本题。

这个问题的解决我们巧妙地应用了线段遮挡问题的性质,即修改一条线段对后面线段的影响是很少的,所以才能设计出数据结构解决本题。

\begin{problem}
  捉迷藏\footnote{ZJOI2007}

  给出一棵 $n$ 个点的树,每个点有黑或白一种颜色。

  有 $m$ 个操作,每个操作改变一个点的颜色,询问最远的黑点对的距离。

  数据范围:$n\leq 10^5, m\leq 5\times 10^5$
\end{problem}

这个题目又是单点修改,我们可以考虑用线段树来解决它。

首先这是一个树上的问题,我们首先用dfs序把它转化成区间问题。

感觉还是不好做,因为树上的距离不好直接在线段上表示出来。

这个时候,我们可以考虑树的一个关于距离的性质:在树的括号序列中,一个点到另一个点的距离恰好等于两点之间删去匹配括号之后剩下的括号的数量。

\newpage

于是对于每个区间,我们维护:

\begin{itemize}
  \item 删去匹配括号之后的左括号、右括号的数量;
  \item 所有黑点到左端点、右端点左、右括号的数量的最大值;
  \item 所有黑点到左端点、右端点左括号数量-右括号数量、右括号数量-左括号数量的最大值。
  \item 该区间中最远黑远点对的距离。
\end{itemize}

怎么从子区间合并上述信息并不难,在此不再赘述。

这个算法的时间复杂度为$O(n\log n)$,可以完美地解决本题。

\subsection{\kai{一类难点在处理标记的线段树问题}}

除了上面这种维护的信息比较复杂的问题外,还有一类修改操作比较特别,需要使用懒标记来解决的问题。

\subsubsection{\kai{从全局出发,一种“操作若干次就不再有效”的问题}}

这是一类通用的问题,其特点在于操作一旦超过若干次之后作用就可以直接处理。

其一般解决方法就是尝试进行全局操作。

\begin{problem}
  Rikka with Phi\footnote{hdu5634}

  给出一个长度为$n$的序列$A$,有$m$个操作。

  1.给出$l,r$,将所有的$A[i],l\leq i\leq r$全部变为$\varphi(A[i])$;

  2.给出$l,r,x$,将所有的$A[i],l\leq i\leq r$全部变为$x$;

  3.给出$l,r$,询问$\sum_{i=l}^rA[i]$

  数据范围:$n,m\leq 3\times 10^5$
\end{problem}

初看觉得十分不可做,但是我们尝试把所有操作看成全局操作:

覆盖操作会导致所有数字变成一样;

取欧拉函数操作最多$\log(x_i)$次就会使所有数变成1。

于是我们维护当前区间的最大值和最小值,如果最大值等于最小值那么直接打上覆盖标记;否则递归处理下面的区间。

我们来分析复杂度。单看欧拉函数操作,每个数最多$\log(x_i)$次就会变成1,复杂度显然是$n\log n$的。

如果加上覆盖操作,根据我们的算法,我们可以把相同的一段区间看作一个数字。

那么每一次覆盖操作最多增加一个数字,均摊下来不会降低取欧拉函数操作的复杂度。

所以总的时间复杂度是$O(n\log n)$。

\subsubsection{\kai{李超线段树}}

\subsubsection{\kai{区间最值操作与历史最值询问}}

\section{\kai{权值线段树的基础应用}}

\subsection{\kai{部分代替平衡树}}

\section{\kai{可持久化线段树的基础应用}}

\subsection{\kai{一类关于子序列的问题}}

\subsection{\kai{一类维护序列中某个区间信息的问题}}

\section{\kai{动态开点线段树的基础应用}}

\section{\kai{线段树合并的基础应用}}

\subsection{\kai{一类考虑子树对父亲的贡献的问题}}

\subsection{\kai{一类有关图的最短路的问题}}

\section{\kai{线段树套线段树的基础应用}}

\section{\kai{线段树作为辅助数据结构的一些问题}}

\subsection{\kai{用线段树优化dp}}

\subsection{\kai{用线段树判断完美匹配}}

\subsection{\kai{用线段树模拟费用流}}

\section{\kai{线段树问题的一些小技巧}}

\subsection{\kai{差分}}

\subsection{\kai{下标有关}}

\end{document}
